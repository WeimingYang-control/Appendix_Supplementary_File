\documentclass[journal,onecolumn]{IEEEtran}
% \usepackage{amsmath,amsfonts}
% \usepackage{algorithmic}
% \usepackage{algorithm}
\usepackage{array}
% \usepackage[caption=false,font=normalsize,labelfont=sf,textfont=sf]{subfig}
% \usepackage{textcomp}
% \usepackage{stfloats}
\usepackage{url}
% \usepackage{verbatim}
% \usepackage{graphicx}
% \usepackage{cite}
% \hyphenation{op-tical net-works semi-conduc-tor IEEE-Xplore}
% % updated with editorial comments 8/9/2021


\usepackage{multirow}
\usepackage{makecell}
\usepackage[portuguese,english]{babel}
\usepackage[utf8]{inputenc}
\usepackage[T1]{fontenc}
\usepackage{booktabs}
\usepackage[dvipsnames,table]{xcolor}
\usepackage{enumitem}
\usepackage{amsmath,amssymb,empheq}
\usepackage{mathtools}
\usepackage{units}
\usepackage{physics} % to use the partial derivative
\usepackage{algorithm}
\usepackage{algpseudocode}
\usepackage{subfigure}
\usepackage{booktabs,paralist}
\usepackage{cite} %enables several citations in brackets
\mathtoolsset{showonlyrefs=false} % to only show referenced equations
\DeclareMathAlphabet\vcal{OMS}{cmsy}{b}{n} % Used for defining \vcal{S}
\usepackage{amsthm} %provides the environment proof 
\usepackage{cuted} % for strip command
\setlength\stripsep{4pt plus 1pt minus 2pt}
\usepackage{multicol}
\usepackage{multirow,hhline}
\usepackage{graphicx}

\newtheorem{property}{Property}
\newtheorem{theorem}{Theorem}
\newtheorem{corollary}{Corollary}
\newtheorem{assumption}{Assumption}
\newtheorem{lemma}{Lemma}
\newtheorem{remark}{Remark}
% 
% revision arrangement
\newcommand{\joel}[1]{{\color{red} #1}}
\newcommand{\weiming}[1]{{\color{Green} #1}}
\newcommand{\minus}{\scalebox{0.75}[1.0]{$-$}}
\newcommand{\tabincell}[2]{\begin{tabular}{@{}#1@{}}#2\end{tabular}}

\usepackage[bookmarks=false,pdftex,pagebackref=false]{hyperref}

\hypersetup{									
	pdfdisplaydoctitle=true,					
	bookmarksnumbered=true,						
	colorlinks,
	citecolor=NavyBlue,
	%hidelinks,
	bookmarksopen=true,
	pdfauthor={Weiming Yang, Gan Yu, Joel Reis, and Carlos Silvestre},
	pdftitle={Robust Nonlinear 3D Control of an Inverted Pendulum Balanced on a Quadrotor: Supplementary Material},
	pdfsubject={},
	pdfkeywords={}
}

\usepackage{mymacros}

\renewcommand{\theequation}{SM \arabic{equation}}

\begin{document}

\title{Nonlinear Output Feedback Control of an Underactuated  Flying Inverted Pendulum: Supplementary Material}

\author{Weiming~Yang,
		Joel~Reis,
        Gan~Yu, and 
        Carlos~Silvestre
        % <-this % stops a space
%    \thanks{The authors are with the Department of Electrical and Computer Engineering, Faculty of Science and Technology, University of Macau, Macau, China (e-mail:\href{mailto:weiming.yang@connect.um.edu.mo}{\{weiming.yang}\href{mailto:ganyu@um.edu.mo}{,ganyu}\href{mailto:joelreis@um.edu.mo}{,joelreis}\href{mailto:csilvestre@um.edu.mo}{,csilvestre\}@um.edu.mo})}%
%	\thanks{C. Silvestre is on leave from Instituto Superior Técnico, Universidade de Lisboa, Lisboa, Portugal.}
	\thanks{Weiming~Yang and Carlos~Silvestre are with the Department of Electrical and Computer Engineering, Faculty of Science and Technology, University of Macau, China (e-mail: weiming.yang@connect.um.edu.mo, csilvestre@um.edu.mo).
		Joel~Reis is with the Department of Ocean Science and Technology at the same institution (e-mail: joelreis@um.edu.mo).}
	\thanks{Gan~Yu is with the School of Sensing Science and Engineering, School of Electronic Information and Electrical Engineering, Shanghai Jiao Tong University, Shanghai 200240, China (e-mail: sjtuganyu@sjtu.edu.cn).}
	\thanks{C. Silvestre is on leave from Instituto Superior Técnico, Universidade de Lisboa, Lisboa, Portugal.}
}

\markboth{Updated last time: \today}%
{Shell \MakeLowercase{\textit{et al.}}: A Sample Article Using IEEEtran.cls for IEEE Journals}

\maketitle

\begin{abstract}
	This is a complementary document to the paper presented in \cite{2024_Yang_TIE}.
	Here we fully disclose all the detailed derivations and equations that are essential to implement the controller reported therein.
\end{abstract}

\section{The transition matrix and the observability Gramian calculations}

The most general transition matrix is given by the Peano–Baker series
	\begin{align}
	\bs{\Phi}(t,t_0):= {\mathbf I} + \int_{t_0}^t{\mathbf A}(\sigma_1)d\sigma_1 + \int_{t_0}^t{\mathbf A}(\sigma_1)\int_{t_0}^{\sigma_1}{\mathbf A}(\sigma_2)d\sigma_2d\sigma_1 +
	\int_{t_0}^t{\mathbf A}(\sigma_1)\int_{t_0}^{\sigma_1}{\mathbf A}(\sigma_2)\int_{t_0}^{\sigma_2}{\mathbf A}(\sigma_3)d\sigma_3d\sigma_2d\sigma_1
	 \cdots
\end{align}

It is noticed that the matrix of ${\mathbf A}$ for the proposed LTV system as shown in (12) \cite{2024_Yang_TIE} is nilpotent of index 3, meaning ${\mathbf A}^n={\mathbf 0}$ for $n\geq3$ where $n$ is a positive integer.
Hence, the proposed LTV system's transition matrix is given by
%
	\begin{align}
	\bs{\Phi}(t,t_0):= {\mathbf I} + \int_{t_0}^t{\mathbf A}(\sigma_1)d\sigma_1 + \int_{t_0}^t{\mathbf A}(\sigma_1)\int_{t_0}^{\sigma_1}{\mathbf A}(\sigma_2)d\sigma_2d\sigma_1 
	\label{def:Phi_t_t0}
\end{align}
%
where
%
\begin{align}
	\int_{t_0}^t{\mathbf A}(\sigma_1)d\sigma_1= 
	&\begin{bmatrix}
			{\bf 0}  & \bs{n_{t}} & {\bf 0} & {\bf 0} & {\bf 0} & {\bf 0}\\
			{\bf 0}  & {\bf 0} & \bs{n_{t}}& {\bf 0} & {\bf 0} & {\bf 0}\\
			{\bf 0}  & {\bf 0} & {\bf 0} & {\bf 0} & {\bf 0} & {\bf 0}\\
			{\bf 0}  & {\bf 0} & {\bf 0} & {\bf 0} & \bs{n_{t}}& {\bf 0}\\
			{\bf 0}  & {\bf 0} & \bs{n_{q}} & {\bf 0} & {\bf 0} & \bs{n_{t}}\\
			{\bf 0}  & {\bf 0} & {\bf 0} & {\bf 0} & {\bf 0} & {\bf 0}
		\end{bmatrix}~\text{and}
\end{align}
%
\begin{align}
	\int_{t_0}^t{\mathbf A}(\sigma_1)\int_{t_0}^{\sigma_1}{\mathbf A}(\sigma_2)d\sigma_2d\sigma_1=
	\begin{bmatrix}
			{\bf 0}  & {\bf 0} & \frac{1}{2}\bs{n_{t}}^2 & {\bf 0} & {\bf 0} & {\bf 0}\\
			{\bf 0}  & {\bf 0} & {\bf 0} & {\bf 0} & {\bf 0} & {\bf 0}\\
			{\bf 0}  & {\bf 0} & {\bf 0} & {\bf 0} & {\bf 0} & {\bf 0}\\
			{\bf 0}  & {\bf 0} & \bs{n_{q2}} & {\bf 0} & {\bf 0} & \frac{1}{2}\bs{n_{t}}^2\\
			{\bf 0}  & {\bf 0} & {\bf 0} & {\bf 0} & {\bf 0} & {\bf 0}\\
			{\bf 0}  & {\bf 0} & {\bf 0} & {\bf 0} & {\bf 0} & {\bf 0}
		\end{bmatrix},
\end{align}
where $\bs{n_{t}}:=(t-t_0){\bf I}$,
% $\bs{n_{t2}}:=\frac{1}{2}(t-t_0)^2{\bf I}$,
  $\bs{n_{q}}:=\frac{\mQ\ell}{\mT}\int_{t_0}^t\dir\dir[T]d\tau$ and
$\bs{n_{q2}}:=\int_{t_0}^t\bs{n_{q}} d\tau$.

%It is noticed that the matrix of ${\mathbf A}$ is nilpotent of index 3, i.e. ${\mathbf A}^n={\mathbf 0}$ for $n\geq3$ where $n$ is a positive integer.
Based on the definition of the observability Gramian and using \eqref{def:Phi_t_t0} along with the nilpotent property of ${\mathbf A}$, it is given by
%
	\begin{align}
		\bs{W}(t_0,t_f)=\int_{t_0}^{t_f}
		\begin{bmatrix}
			\v{I}  & \bs{n_{t}} & \frac{1}{2}\bs{n_{t}}^2  & {\bf 0} & {\bf 0} & {\bf 0} \\
			\bs{n_{t}} & \bs{n_{t}}^2 & \frac{1}{2}\bs{n_{t}}^3 & {\bf 0} & {\bf 0} & {\bf 0} \\
			\frac{1}{2}\bs{n_{t}}^2 & \frac{1}{2}\bs{n_{t}}^3 & \frac{1}{4}\bs{n_{t}}^4+\bs{n_{q2}}\trs\bs{n_{q2}} & \bs{n_{q2}}\trs & \bs{n_{t}}\bs{n_{q2}}\trs & \frac{1}{2}\bs{n_{t}}^2\bs{n_{q2}}\trs\\
			{\bf 0} & {\bf 0} & \bs{n_{q2}}  & \v{I}  & \bs{n_{t}} & \frac{1}{2}\bs{n_{t}}^2\\
			{\bf 0} & {\bf 0} & \bs{n_{t}}\bs{n_{q2}} & \bs{n_{t}} & \bs{n_{t}}^2 & \frac{1}{2}\bs{n_{t}}^3\\
			{\bf 0} & {\bf 0} & \frac{1}{2}\bs{n_{t}}^2\bs{n_{q2}} & \frac{1}{2}\bs{n_{t}}^2 & \frac{1}{2}\bs{n_{t}}^3 & \frac{1}{4}\bs{n_{t}}^2
		\end{bmatrix}~dt
		\label{eq:W_t0_tf}
	\end{align}

Setting $t_0=t-\delta$ and $t_f=t$, $\bs{W}(t_0,t_f)$ is obtained as follows
%
	\begin{align}
		\bs{W}(t-\delta,t)=
		\begin{bmatrix}
			\delta\v{I}  & \frac{\delta^2}{2}\v{I} & \frac{\delta^3}{6}\v{I}  & {\bf 0} & {\bf 0} & {\bf 0} \\
			\frac{\delta^2}{2}\v{I} & \frac{\delta^3}{3}\v{I} & \frac{\delta^4}{8}\v{I} & {\bf 0} & {\bf 0} & {\bf 0} \\
			\frac{\delta^3}{6}\v{I} & \frac{\delta^4}{8}\v{I} & \frac{\delta^5}{20}\v{I}+\int_{t-\delta}^{t}\bs{n_{q2}}\trs\bs{n_{q2}}d\tau & \int_{t-\delta}^{t}\bs{n_{q2}}\trs d\tau & \bs{\varphi}_1\trs & \bs{\varphi}_2\trs\\
			{\bf 0} & {\bf 0} & \int_{t-\delta}^{t}\bs{n_{q2}}d\tau  & \delta\v{I}  & \frac{\delta^2}{2}\v{I} & \frac{\delta^3}{6}\v{I}\\
			{\bf 0} & {\bf 0} & \bs{\varphi}_1 & \frac{\delta^2}{2}\v{I} & \frac{\delta^3}{3}\v{I} & \frac{\delta^4}{8}\v{I}\\
			{\bf 0} & {\bf 0} & \bs{\varphi}_2 & \frac{\delta^3}{6}\v{I} & \frac{\delta^4}{8}\v{I} & \frac{\delta^5}{20}\v{I}
		\end{bmatrix}
		\label{eq:W_t_delta}
	\end{align}
%
%(t-t_0)\bs{n_{q2}}
%n_t\bs{n_{q2}}
where $\bs{\varphi}_1:=\frac{\delta^2}{2}\bs{n_{q2}} - \frac{\delta^2}{2} \int_{t-\delta}^{t}\bs{n_{q}}d\tau$ and $\bs{\varphi}_2:=\frac{\delta^3}{6}\bs{n_{q2}} - \frac{\delta^3}{6} \int_{t-\delta}^{t}\bs{n_{q}}d\tau$.

Since $\dir$ is a unit vector, each element in \eqref{eq:W_t_delta} is bounded for some $\delta$.
Consequently, $\bs{W}(t-\delta,t)$ is bounded as well.

\begin{remark}
	When $\dir$ remains constant, such as when it equals $-\ez$, $\bs{n_{q}}$ tends to zero as $t$ approaches infinity.
	However, in reality, it is impossible to maintain $\dir$ constantly at $-\ez$ throughout all FIP maneuvers, even for a constant position tracking.
\end{remark}

\section{Proof of Theorem 3}\label{sec:x_tilde}
%
Based on $\bs{x}(t)$ (10) and $\widehat{\bs{x}}(t)$ (19) in \cite{2024_Yang_TIE}, the estimation error is defined as $\widetilde{\bs{x}}(t):=\bs{x}(t)-\widehat{\bs{x}}(t)$.
%
%\begin{theorem}
%		Granted the continuous time LTV system (11) in \cite{2024_Yang_TIE} is u.c.o., and $\bs{\mathcal{P}} (t_0)\in\mathbb{R}^{18\times18}_{\succ0}$, $\widetilde{\bs{x}}$ is uniformly ultimately bounded.
%\end{theorem}
%
%\begin{proof}
Using $\dot{{\bs{x}}}(t)$ in (11) and $\dot{\widehat{\bs{x}}}(t)$ in (18) from \cite{2024_Yang_TIE}, we obtain the time derivative of the estimation error $\widetilde{\bs{x}}(t)$ as:
%
\begin{equation}
	\dot{\widetilde{\bs{x}}}(t)=({\bf A}(t)-\bs{\mathcal{K}} (t){\bf C})\widetilde{\bs{x}}(t) +\bs{h}(t).
\end{equation}
%
Considering the Lyapunov function $V_{\text{KF}}:=\widetilde{\bs{x}}\trs\bs{\mathcal{P}}^{-1}\widetilde{\bs{x}}$, 
where $\bs{\mathcal{P}}(t)$ denotes the covariance matrix of the state estimate, and Assumption made in \cite{2024_Yang_TIE}, we can derive the following inequality for the time derivative of $V_{\text{KF}}$:
%
\begin{equation}
	\lambda_{\min}(\bs{\mathcal{P}}^{-1})\|\widetilde{\bs{x}}\|^2 \leq V_{\text{KF}} \leq \lambda_{\max}(\bs{\mathcal{P}}^{-1})\|\widetilde{\bs{x}}\|^2
\end{equation}
%
where $\lambda_{\max}(\bs{\mathcal{P}}^{-1})\geq\lambda_{\min}(\bs{\mathcal{P}}^{-1})>0$.

Using (18), the time derivative of $V_{\text{KF}}$ can be written as
%
\begin{equation}
	\dot{V}_{\text{KF}}\leq-\lambda_{\min}(\bs{\Upsilon}(t))\|\widetilde{\bs{x}}(t)\|^2+2\widetilde{\bs{x}}\trs(t)\bs{\mathcal{P}}^{-1}(t)\bs{h}(t)
\end{equation}
% 
where $\bs{\Upsilon}(t):={\mathbf C}\trs \bs{\mathcal{R}} ^{-1}(t){\mathbf C}+\bs{\mathcal{P}}^{-1}(t)\bs{\mathcal{Q}} (t)\bs{\mathcal{P}}^{-1}(t)\in\mathbb{R}^{18\times18}_{\succ0}$.
Notice that $\|\partial V_{\text{KF}}/\partial \widetilde{\bs{x}}\|\leq 2\lambda_{\min}(\bs{\mathcal{P}}^{-1})\|\widetilde{\bs{x}}\|$.

Based on Assumption~1, suppose the perturbation term $\bs{h}(t)$ satisfies $\|\bs{h}(t)\| \leq \varepsilon <\frac{\lambda_{\min}(\bs{\Upsilon})}{2\lambda_{\min}(\bs{\mathcal{P}}^{-1})}\sqrt{\frac{\lambda_{\min}(\bs{\mathcal{P}}^{-1})}{\lambda_{\max}(\bs{\mathcal{P}}^{-1})}}\epsilon r$, for all $t\geq0$, for all $\widetilde{\bs{x}}\in D$ where $D=\{ \widetilde{\bs{x}}\in\mathbb{R}^{18}| \|\widetilde{\bs{x}}\|<r \}$, and for some positive constant $\epsilon<1$.

According to Lemma 9.2 in \cite{2002_Khalil}, for all $\widetilde{\bs{x}}(t_0)<\sqrt{\frac{\lambda_{\min}(\bs{\mathcal{P}}^{-1})}{\lambda_{\max}(\bs{\mathcal{P}}^{-1})}}r$, and some finite $T$, the solution of $\widetilde{\bs{x}}(t)$ satisfies
%
\begin{equation}
	\|\widetilde{\bs{x}}(t)\|\leq \sqrt{\frac{\lambda_{\max}(\bs{\mathcal{P}}^{-1})}{\lambda_{\min}(\bs{\mathcal{P}}^{-1})}}\exp(-\frac{(1-\epsilon)\lambda_{\min}(\bs{\Upsilon})}{2\lambda_{\max}(\bs{\mathcal{P}}^{-1})}(t-t_0))\|\widetilde{\bs{x}}(t_0)\|, \quad\forall t_0\leq t<t_0+T
\end{equation}
%
and
%
\begin{equation}
	\|\widetilde{\bs{x}}(t)\|\leq \frac{2\lambda_{\min}(\bs{\mathcal{P}}^{-1})}{\lambda_{\min}(\bs{\Upsilon})} \sqrt{\frac{\lambda_{\max}(\bs{\mathcal{P}}^{-1})}{\lambda_{\min}(\bs{\mathcal{P}}^{-1})}}\frac{\varepsilon}{\epsilon}
\end{equation}


\section{Computation of auxiliary variables}
\label{sec:aux_notation}

	We start by noting that (4a) may be rewritten as
	%
	\begin{equation}
		\vLdot = \vLdot|_{\f=\fd^{\|}} + \frac{1}{\mT}(\f^{\|}-\fd^{\|}) + \widetilde{\f_v}^{\|} + \frac{\mQ\ellp}{\mT}\widetilde{\f_q}^{\|},
		\label{eq:vL_dot_aux1}
	\end{equation}
	%
	where
	%
	\begin{equation}
		\vLdot|_{\f=\fd^{\|}} = g\ez - \frac{1}{\mT}\widehat{\Xiq}^{\|}.
	\end{equation}

	All the time derivatives that explicitly feature $\vLdot$ can be divided into three terms: one related to $\fd^{\|}$, one related to the error $\f^{\|}-\fd^{\|}$, and another related to $\widetilde{\f_v}$ and $\widetilde{\f_q}$.

	Similarly, we can rearrange (8) as
	%
	\begin{equation}
		\zvdot = \zvdot|_{\f=\fd^{\|}} + \frac{1}{\mT}\frac{\partial\zv}{\partial\vL}(\f^{\|}-\fd^{\|}) + \frac{1}{\mT}\frac{\partial\zv}{\partial\vL}(\widetilde{\f_v}^{\|} + \frac{\mQ\ellp}{\mT}\widetilde{\f_q}^{\|}),
		\label{eq:zv_dot_aux1}
	\end{equation}
	%
	where
	%
	\begin{equation}
		\zvdot|_{\f=\fd^{\|}} = \vLdot|_{\f=\fd^{\|}} - \pLdddot
	\end{equation}
	%
	and
	\begin{equation}
		\frac{\partial\zv}{\partial\vL} = \v{I}.
	\end{equation}

Then
%
\begin{align}
	\dot{\Xiq} = \dot{\Xiq}|_{\f=\fd^{\|}} + \frac{\partial \Xiq}{\partial \vL}\LP\frac{1}{\mT}(\f^{\|}-\fd^{\|}) + \widetilde{\f_v}^{\|} + \frac{\mQ\ellp}{\mT}\widetilde{\f_q}^{\|}\RP,
\end{align}
%
where
%
\begin{align}
	\dot{\Xiq}|_{\f=\fd^{\|}} = \mT\LP {\bf K_p}\zv + {\bf K_v} \zvdothat|_{\f=\fd^{\|}}
	- \pLdtdot\RP,
\end{align}
%
its corresponding estimation
%
\begin{align}
	\widehat{\dot{\Xiq}}|_{\f=\fd^{\|}} = \mT\LP {\bf K_p}\zvhat + {\bf K_v} \zvdothat|_{\f=\fd^{\|}}
	- \pLdtdot\RP
\end{align}
%
and $\frac{\partial \Xiq}{\partial \vL}:=\mT{\bf K_v}$.

	We start by presenting the expression for $\widehat{\dot{\bs{\varpi_d}}}|_{\f=\fd^{\|}}$, first shown in (29).
    We have
    %
    \begin{align}
    	\widehat{\dot{\bs{\varpi_d}}}|_{\f=\fd^{\|}} =& \frac{1}{\|\widehat{\Xiq}\|}
    	\Skew{\widehat{\bs{\varpi}}}\Skew{\qdhat}\widehat{\dot{\Xiq}}|_{\f=\f_d^{\|}}
    	-\Skew{\dir}\Skew{\frac{1}{\|\widehat{\Xiq}\|}\widehat{\dot{\Xiq}}|_{\f=\f_d^{\|}}}\qddothat|_{\f=\f_d^{\|}} +\Skew{\dir}\Skew{\qdhat}\LP\frac{1}{\|\widehat{\Xiq}\|}\frac{d}{dt}\LP\widehat{\dot{\Xiq}}|_{\f=\fd^{\|}}\RP_{est}
    	-\frac{1}{\|\widehat{\Xiq}\|^3}\widehat{\dot{\Xiq}}|_{\f=\f_d^{\|}}\widehat{\Xiq}\trs\widehat{\dot{\Xiq}}|_{\f=\f_d^{\|}}\RP\nonumber\\ &-\frac{k_\v{q}}{h_\v{q}}\LP\Skew{\widehat{\bs{\varpi}}}\Skew{\dir}\qdhat +\Skew{\dir}\Skew{\widehat{\bs{\varpi}}}\qdhat +\Skew[2]{\dir}\qddothat|_{\f=\f_d^{\|}}\RP,
    \end{align}
    %
%    %
%    \begin{align}
%        \zwdothat|_{\f=\fd^{\|}} =~& \Skew{\dirdot}\bs{\omega} -\frac{k_q}{h_{q}}\LP\Skew{\dirdot}\Skew{\dir}\dird +\Skew{\dir}\Skew{\dirdot}\dird +\Skew[2]{\dir}\qddothat|_{\f=\f_d^{\|}}\RP
%         + \Skew{\dirdot}\Skew{\dird}\frac{1}{\|\Xiq\|}\widehat{\dot{\Xiq}}|_{\f=\f_d^{\|}} \nonumber \\
%         &-\frac{1}{\|\Xiq\|}\Skew{\dir}\Skew{\widehat{\dot{\Xiq}}|_{\f=\f_d^{\|}}}\qddothat|_{\f=\f_d^{\|}}  +\Skew{\dir}\Skew{\dird}\left(\frac{\frac{d}{dt}\LP\widehat{\dot{\Xiq}}|_{\f=\fd^{\|}}\RP_{\text{est}}|_{\f=\fd^{\|}}}{\|\Xiq\|}
%         -\widehat{\dot{\Xiq}}|_{\f=\f_d^{\|}}\frac{\Xiq\trs}{\|\Xiq\|^3}\widehat{\dot{\Xiq}}|_{\f=\f_d^{\|}}\right)\nonumber\\
%        &+ \frac{\epsilon}{h_{q}\mT}\|\Xiq\|\left(\vphantom{\frac{\Xiq\trs}{\|\Xiq\|}} \LP\Skew{\dirdot}\Skew{\dir}+\Skew{\dir}\Skew{\dirdot}\RP{\bf e} + \frac{1}{\|\Xiq\|^2}\Skew[2]{\dir}{\bf e}\Xiq\trs\widehat{\dot{\Xiq}}|_{\f=\fd^{\|}} + \Skew[2]{\dir}\widehat{\dot{\bf e}}|_{\f=\fd^{\|}}\right),
%        \label{eq:lambda}
%    \end{align}
    %
    where
%    %
%    \begin{equation}
%        \widehat{\dot{\bf e}}|_{\f=\fd^{\|}} =\epsilon^2{\bf K_p}\zv + \epsilon\zvdothat|_{\f=\fd^{\|}},
%    \end{equation}
%	%
%	and
%	%
%	\begin{equation}
%		\widehat{\dot{\Xiq}}|_{\f=\fd^{\|}} = \mT\LP \widehat{\dot{\bs{\zeta}}}|_{\f=\fd^{\|}}
%		- \pLdtdot\RP,
%	\end{equation}
%	%
%	with
%	%
%	\begin{equation}
%		\widehat{\dot{\bs{\zeta}}}|_{\f=\fd^{\|}} = \epsilon^2( {\bf I} + {\bf K_v}{\bf K_p} )\zv + \epsilon( {\bf K_p} + {\bf K_v} )\zvdothat|_{\f=\fd^{\|}},
%	\end{equation}
%	%
%	and where
	%
	\begin{equation}
		\qddothat|_{\f=\f_d^{\|}} = \frac{1}{\|\Xiq\|}\Skew[2]{\qdhat}\widehat{\dot{\Xiq}}|_{\f=\f_d^{\|}},
		\label{eq:qd_dot_fd}
	\end{equation}
	%
	\begin{equation}
		\frac{d}{dt}\LP\zvdothatdot|_{\f=\fd^{\|}}\RP_{\text{est}}|_{\f=\fd^{\|}} =  \minus\frac{1}{\mT}\LSB(\widehat{\bs{\varpi}}\dir[T] + \dir\widehat{\bs{\varpi}}\trs)\widehat{\Xiq} + \dir\dir[T]\widehat{\dot{\Xiq}}|_{\f=\fd^{\|}}\RSB - \pLdtdot,
	\end{equation}
	%
%	\begin{equation}
%		\frac{d}{dt}\LP\widehat{\dot{\bs{\zeta}}}|_{\f=\fd^{\|}}\RP_{\text{est}}|_{\f=\fd^{\|}} = \epsilon^2( {\bf I} + {\bf K_v}{\bf K_p} )\zvdothat|_{\f=\fd^{\|}}
%		+ \epsilon( {\bf K_p} + {\bf K_v} )\frac{d}{dt}\LP\zvdothatdot|_{\f=\fd^{\|}}\RP_{\text{est}}|_{\f=\f_d^{\|}},
%	\end{equation}
	%
	and, finally,
	%
	\begin{equation}
		\frac{d}{dt}\LP\widehat{\dot{\Xiq}}|_{\f=\fd^{\|}}\RP_{\text{est}}|_{\f=\fd^{\|}} = \mT\left( {\bf K_p} \zvdothat|_{\f=\fd^{\|}}
		+  {\bf K_v} \frac{d}{dt}\LP\zvdothatdot|_{\f=\fd^{\|}}\RP_{\text{est}}|_{\f=\f_d^{\|}}
		- \pLdqdot\right).
	\end{equation}
	
	The expression for $\partial\zc/\partial\vL$, also first shown in (29), is given by
    %
    \begin{align}
        \left.\frac{\partial\zc}{\partial\vL}\right|_{\text{est}} =&~ \minus \frac{k_q}{h_{q}}\Skew[2]{\dir}\left.\frac{\partial\dird}{\partial\vL}\right|_{\text{est}} -
        \Skew{\dir}\Skew{\frac{\widehat{\dot{\Xiq}}|_{\f=\f_d^{\|}}}{\|\widehat{\Xiq}\|}} \left.\frac{\partial\dird}{\partial\vL}\right|_{\text{est}}
        -\frac{\Skew{\dir}\Skew{\qdhat}}{\|\widehat{\Xiq}\|^2}\left(\|\widehat{\Xiq}\|\frac{\partial\dot{\Xiq}|_{\f=\fd^{\|}}}{\partial\vL} + \frac{\widehat{\dot{\Xiq}}|_{\f=\f_d^{\|}}\widehat{\Xiq}\trs}{\|\widehat{\Xiq}\|}\frac{\partial\Xiq}{\partial\vL}\right),
        \label{eq:partial_zw_vL}
	\end{align}
	%
	where
	%
	\begin{equation}
		\left.\frac{\partial\dird}{\partial\vL}\right|_{\text{est}} = \frac{1}{\|\widehat{\Xiq}\|}\Skew[2]{\qdhat}\frac{\partial\Xiq}{\partial\vL},
	\end{equation}
	%
	with
	\begin{equation}
		\frac{\partial\Xiq}{\partial\vL} = \mT\v{K}_\v{v}\frac{\partial\zv}{\partial\vL},
	\end{equation}
	%
	and where
	%
	%
	\begin{equation}
		\frac{\partial\zvdot|_{\f=\fd^{\|}}}{\partial\vL}=-\frac{1}{\mT}\dir\dir[T]\frac{\partial\Xiq}{\partial\vL},
	\end{equation}
	%
	%
	\begin{equation}
		\frac{\partial\dot{\Xiq}|_{\f=\fd^{\|}}}{\partial\vL}=\mT\LP\frac{1}{\mT}{\bf K_p} {\bf I}^{\|} + {\bf K_v} \frac{\partial\zvdot|_{\f=\fd^{\|}}}{\partial\vL}\RP.
	\end{equation}
%	%
%	and, finally,
%	%
%	\begin{equation}
%		\frac{\partial\qddothat|_{\f=\fd^{\|}}}{\partial\vL} = \frac{1}{\|\Xiq\|}\LP\dird\dot{\Xiq}|_{\f=\fd^{\|}}^{\trs}-\dot{\Xiq}|_{\f=\fd^{\|}}\dird[T]\RP\frac{\partial\dird}{\partial\vL}
%		-\frac{\Skew[2]{\dird}\dot{\Xiq}|_{\f=\fd^{\|}}\Xiq[T]\frac{\partial\Xiq}{\partial\vL}}{\|\Xiq\|^3} +\frac{\Skew[2]{\dird} \frac{\partial\dot{\Xiq}|_{\f=\fd^{\|}}}{\partial\vL}}{\|\Xiq\|}.
%		\label{eq:partial_qd_vL}
%	\end{equation}
%
%	In the following we present some useful partial derivatives of the Lyapunov function candidates with respect to $\vL$ and $\wL$.
%	They are given by
%	%
%	\begin{equation}
%		\frac{\partial V_3}{\partial\vL} = {\bf e}\trs\frac{\partial{\bf e}}{\partial\vL} -  h_{q}\zdir[T]\frac{\partial\dird}{\partial\vL},
%		\label{eq:partial_V3_vL}
%	\end{equation}
%	%
%	\begin{equation}
%		\frac{\partial V_4}{\partial\vL} = \frac{\partial V_3}{\partial\vL} + h_{\omega}\zw[T]\frac{\partial\zw}{\partial\vL},
%		\label{eq:partial_V4_vL}
%	\end{equation}
%	%
%	\begin{equation}
%		\frac{\partial V_5}{\partial\vL} =\frac{\partial V_4}{\partial\vL} - h_{r}\rzd[T]\RotM\Skew{\ez}\RotM[T]\Skew{\rzd}\frac{\partial\rzd}{\partial\vL} ,
%		\label{eq:partial_V5_vL}
%	\end{equation}
%	%
%	and
%	%
%	\begin{equation}
%		\frac{\partial V_5}{\partial\wL} = h_{\omega}\zw[T]\frac{\partial\zw}{\partial\wL}- h_{r}\rzd[T]\RotM\Skew{\ez}\RotM[T]\Skew{\rzd}\frac{\partial\rzd}{\partial\wL}, 
%		\label{eq:partial_V5_wL}
%	\end{equation}
%	%
%	where
%	%
%	\begin{equation}
%		\frac{\partial\zw}{\partial\wL} = \Skew{\dir},
%	\end{equation}
%	%
%	\begin{equation}
%		\frac{\partial\dirdot}{\partial\wL} = \minus\Skew{\dir},
%	\end{equation}
%	%
%	\begin{equation}
%		\frac{\partial\frac{d}{dt}\LP\zvdothatdot|_{\f=\fd^{\|}}\RP_{\text{est}}}{\partial\wL} = \minus\frac{1}{\mT}\LP\dir[T]\Xiq+\dir\Xiq[T]\RP\frac{\partial\dirdot}{\partial\wL},
%	\end{equation}
%	%
%	\begin{equation}
%		\frac{\partial\frac{d}{dt}\LP\zvdothatdot|_{\f=\fd^{\|}}\RP_{\text{est}}}{\partial\vL} = \minus\frac{(\dirdot\dir[T]+\dir\dirdot[T])\frac{\partial\Xiq}{\partial\vL}+\dir\dir[T]\frac{\partial\widehat{\dot{\Xiq}}|_{\f=\fd^{\|}}}{\partial\vL}}{\mT},
%	\end{equation}
%	%
%	\begin{equation}
%		\frac{\partial\frac{d}{dt}\LP\widehat{\dot{\bs{\zeta}}}|_{\f=\fd^{\|}}\RP_{\text{est}}}{\partial\wL} = \epsilon(\v{K}_\v{p} + \v{K}_\v{v} )\frac{\partial\frac{d}{dt}\LP\zvdothatdot|_{\f=\fd^{\|}}\RP_{\text{est}}}{\partial\wL},
%	\end{equation}
%	%
%	\begin{equation}
%		\frac{\partial\frac{d}{dt}\LP\widehat{\dot{\bs{\zeta}}}|_{\f=\fd^{\|}}\RP_{\text{est}}}{\partial\vL} = \epsilon^2( {\bf I} + {\bf K_v}{\bf K_p})\frac{\partial\zvdothat|_{\f=\fd^{\|}}}{\partial\vL} + \epsilon( {\bf K_p} + {\bf K_v} )\frac{\partial\frac{d}{dt}\LP\zvdothatdot|_{\f=\fd^{\|}}\RP_{\text{est}}}{\partial\vL},
%	\end{equation}
%	%
%	\begin{equation}
%		\frac{\partial\frac{d}{dt}\LP\widehat{\dot{\Xiq}}|_{\f=\fd^{\|}}\RP_{\text{est}}}{\partial\wL} = \mT\frac{\partial\frac{d}{dt}\LP\widehat{\dot{\bs{\zeta}}}|_{\f=\fd^{\|}}\RP_{\text{est}}}{\partial\wL},
%	\end{equation}
%	%
%	\begin{equation}
%		\frac{\partial\frac{d}{dt}\LP\widehat{\dot{\Xiq}}|_{\f=\fd^{\|}}\RP_{\text{est}}}{\partial\vL} = \mT\frac{\partial\frac{d}{dt}\LP\widehat{\dot{\bs{\zeta}}}|_{\f=\fd^{\|}}\RP_{\text{est}}}{\partial\vL},
%	\end{equation}
%
%    \begin{align}
%        \frac{\partial\zwdothat|_{\f=\fd^{\|}}}{\partial\vL}&=-\frac{k_q}{h_q}\left(\LSB\Skew{\dirdot}\Skew{\dir}+\Skew{\dir}\Skew{\dirdot}\RSB\frac{\partial\dird}{\partial\vL}
%        +\Skew[2]{\dir}\frac{\partial\qddothat|_{\f=\fd^{\|}}}{\partial\vL}\right)-\Skew{\dirdot}\Skew{\frac{\widehat{\dot{\Xiq}}|_{\f=\fd^{\|}}}{\|\Xiq\|}}\frac{\partial\dird}{\partial\vL} \nonumber\\
%        &
%        +\frac{\Skew{\dirdot}\Skew{\dird}+\Skew{\dir}\Skew{\qddothat|_{\f=\f_d^{\|}}}}{\|\Xiq\|}\LP\frac{\partial\widehat{\dot{\Xiq}}|_{\f=\fd^{\|}}}{\partial\vL}-\frac{\widehat{\dot{\Xiq}}|_{\f=\fd^{\|}}\Xiq[T]\frac{\partial\Xiq}{\partial\vL}}{\|\Xiq\|^2}\RP -\Skew{\dir}\Skew{\frac{\widehat{\dot{\Xiq}}|_{\f=\fd^{\|}}}{\|\Xiq\|}} \frac{\partial\qddothat|_{\f=\fd^{\|}}}{\partial\vL}\nonumber\\
%        &
%        -\Skew{\dir}\Skew{\frac{\|\Xiq\|^2\frac{d}{dt}\LP\widehat{\dot{\Xiq}}|_{\f=\fd^{\|}}\RP_{\text{est}}-\widehat{\dot{\Xiq}}|_{\f=\fd^{\|}}\Xiq[T]\widehat{\dot{\Xiq}}|_{\f=\fd^{\|}}}{\|\Xiq\|^3}}\frac{\partial\dird}{\partial\vL} +\Skew{\dir}\Skew{\dird}\left(-\frac{\Xiq[T]\widehat{\dot{\Xiq}}|_{\f=\fd^{\|}}\frac{\partial\widehat{\dot{\Xiq}}|_{\f=\fd^{\|}}}{\partial\vL}-
%        \widehat{\dot{\Xiq}}|_{\f=\fd^{\|}}\Xiq[T]\frac{\partial\widehat{\dot{\Xiq}}|_{\f=\fd^{\|}}}{\partial\vL}}{\|\Xiq\|^3}\right.\nonumber\\
%        &
%        \left.\frac{\frac{\partial\frac{d}{dt}\LP\widehat{\dot{\Xiq}}|_{\f=\fd^{\|}}\RP_{\text{est}}}{\partial\vL}\|\Xiq\|-\frac{d}{dt}\LP\widehat{\dot{\Xiq}}|_{\f=\fd^{\|}}\RP_{\text{est}}\Xiq[T]\frac{\partial\Xiq}{\partial\vL}}{\|\Xiq\|^2}
%        +\frac{\widehat{\dot{\Xiq}}|_{\f=\fd^{\|}}\LP\|\Xiq\|^3\widehat{\dot{\Xiq}}|_{\f=\fd^{\|}}\trs\frac{\partial\Xiq}{\partial\vL}-3\|\Xiq\|\Xiq[T]\widehat{\dot{\Xiq}}|_{\f=\fd^{\|}}\Xiq[T]\frac{\partial\Xiq}{\partial\vL}\RP}{\|\Xiq\|^6}\right),
%    \end{align}
%	%
%	and, at last,
%	%
%    \begin{align}
%        \frac{\partial\zwdothat|_{\f=\fd^{\|}}}{\partial\wL}=&-\Skew{\wL}\frac{\partial\dirdot}{\partial\wL}+\Skew{\dirdot}-\frac{k_q}{h_q}(\dir\dird[T]-\dird\dir[T])\frac{\partial\dirdot}{\partial\wL}
%        +\frac{k_q}{h_q}\Skew{\dir}\Skew{\dird}\frac{\partial\dirdot}{\partial\wL} + \frac{\|\Xiq\|\epsilon}{h_{q}\mT}\left((\dir{\bf e}\trs-{\bf e}\dir[T])\frac{\partial\dirdot}{\partial\wL}\right. \nonumber\\
%        &\left.-\Skew{\dir}\Skew{\bf e}\frac{\partial\dirdot}{\partial\wL}\right)+\frac{\dird\widehat{\dot{\Xiq}}|_{\f=\fd^{\|}}\trs-\widehat{\dot{\Xiq}}|_{\f=\fd^{\|}}\dird[T]}{\|\Xiq\|}\frac{\partial\dirdot}{\partial\wL}
%        +\frac{1}{\|\Xiq\|}\Skew{\dir}\Skew{\dird}\frac{\partial\frac{d}{dt}\LP\widehat{\dot{\Xiq}}|_{\f=\fd^{\|}}\RP_{\text{est}}}{\partial\wL},
%    \end{align}
%	
%	Other important partial derivatives used in the derivations are the following:
%	%
%	\begin{equation}
%		\frac{\partial\fd}{\partial\vL} = \minus\dir\dir[T]\frac{\partial\Xiq}{\partial\vL} + \mQ\ellp\Skew[2]{\dir}\left(\frac{\partial\zwdothat|_{\f=\fd^{\|}}}{\partial\vL} + \frac{h_q}{h_{\omega}}\frac{\partial\dird}{\partial\vL}\right),
%	\end{equation}
%	%
%	\begin{equation}
%		\frac{\partial\fd}{\partial\wL}=2\mQ\ellp\dir\wL[T]{\bf I}+\mQ\ellp\Skew[2]{\dir}
%	\LP\frac{\partial\zwdothat|_{\f=\fd^{\|}}}{\partial\wL}+\frac{k_{\omega}}{h_{\omega}}\Skew{\dir}\RP,
%	\end{equation}
%	%
%	\begin{equation}
%		\frac{\partial\rzd}{\partial\wL} = \minus\frac{\|\fd\|{\bf I} - \frac{1}{\|\fd\|}\fd\fd[T]}{\|\fd\|^2}\frac{\partial\fd}{\partial\wL},
%	\end{equation}
%	%
%	and
%	%
%	\begin{equation}
%		\frac{\partial\rzd}{\partial\vL} = \minus\frac{\|\fd\|{\bf I}-\frac{1}{\|\fd\|}\fd\fd[T]}{\|\fd\|^2}\frac{\partial\fd}{\partial\vL}.
%	\end{equation}
	
	We also need to compute the pseudo-estimate of $\rzddot$, which is given by
	% 
	\begin{equation}
		\rzddothat = \minus\frac{1}{\|\fd\|^2}\LP\|\fd\|\fddothat - \frac{1}{\|\fd\|}\fd\fd[T]\fddothat\RP,
		\label{eq:rzddothat}
	\end{equation}
	%
	where
	%
	\begin{equation}
		\fddothat = \widehat{\fddot^{\|}}+\widehat{\fddot^{\perp}},
	\end{equation}
	%
	with
	%
	\begin{equation}
		\widehat{\fddot^{\|}} = -\widehat{\bs{\varpi}}\dir[T]\widehat{\Xiq}-\dir\widehat{\bs{\varpi}}\trs\widehat{\Xiq}-\dir\dir[T]\dot{\widehat{\Xiq}}|_{\text{est}}
		-\mT\LP \widehat{\bs{\varpi}}\dir[T]\widehat{\f_v}-\dir\widehat{\bs{\varpi}}\trs\widehat{\f_v}-\dir\dir[T]\dot{\widehat{\f_v}}\RP
		-\mQ\ell\LP \widehat{\bs{\varpi}}\dir[T]\widehat{\f_q}-\dir\widehat{\bs{\varpi}}\trs\widehat{\f_q}-\dir\dir[T]\dot{\widehat{\f_q}}\RP
	\end{equation}
	%
	and
	%
	\begin{equation}
		\begin{aligned}
		\widehat{\fddot^{\perp}} =~& \mQ\ell\Skew{\widehat{\bs{\varpi}}}\Skew{\dir}\LP\widehat{\dot{\bs{\varpi_d}}}|_{\f=\fd^{\|}} + \frac{h_{\dir}}{h_{\bs{\varpi}}}\qdhat -\frac{k_{\bs{\varpi}}}{h_{\bs{\varpi}}}\zchat- \widehat{\f_q}\RP+\mQ\ell\Skew{\dir}\Skew{\widehat{\bs{\varpi}}}\LP\widehat{\dot{\bs{\varpi_d}}}|_{\f=\fd^{\|}} + \frac{h_{\dir}}{h_{\bs{\varpi}}}\qdhat -\frac{k_{\bs{\varpi}}}{h_{\bs{\varpi}}}\zchat- \widehat{\f_q}\RP\nonumber\\
		&+\mQ\ell\Skew[2]{\dir}\LP\frac{d}{dt}\LP\widehat{\dot{\bs{\varpi_d}}}|_{\f=\fd^{\|}}\RP_{\text{est}} +  \frac{h_{\dir}}{h_{\bs{\varpi}}}\frac{d}{dt}\LP\qdhat\RP_{\text{est}} -\frac{k_{\bs{\varpi}}}{h_{\bs{\varpi}}}\frac{d}{dt}\LP\zchat\RP_{\text{est}}- \dot{\widehat{\f_q}}\RP,
		\end{aligned}
	\end{equation}
	%
	where,
	%
	\begin{equation}
		\vLdothat = \frac{1}{\mT}\f^{\|} + \widehat{\f_v} + \frac{\mQ\ellp}{\mT}\widehat{\f_q}^{\|}+g\ez,
	\end{equation}
	%
	\begin{equation}
		\zvdothat = \vLdothat - \pLdddot,
	\end{equation}
	%
%	\begin{equation}
%		\widehat{\dot{\bf e}} = \epsilon^2{\bf K_p}\zv + \epsilon\zvdothat,
%	\end{equation}
	%
%	\begin{equation}
%		\widehat{\dot{\bs{\zeta}}} = \epsilon^2( {\bf I} + {\bf K_v}{\bf K_p} )\zv + \epsilon( {\bf K_p} + {\bf K_v} )\zvdothat,
%	\end{equation}
	%
	\begin{equation}
		\widehat{\dot{\Xiq}} = \mT\LP{\bf K_p} \zvhat +  {\bf K_v} \zvdothat - \pLdtdot\RP,
	\end{equation}
	%
	\begin{equation}
		\frac{d}{dt}\LP\qdhat\RP_{\text{est}} = \frac{1}{\|\widehat{\Xiq}\|}\Skew[2]{\qdhat}\widehat{\dot{\Xiq}}
	\end{equation}
	\begin{equation}
		\frac{d}{dt}\LP\zvdot|_{\f=\fd^{\|}}\RP_{\text{est}} = \minus\frac{(\widehat{\bs{\varpi}}\dir[T] + \dir\widehat{\bs{\varpi}}\trs)\widehat{\Xiq} + \dir\dir[T]\widehat{\dot{\Xiq}}}{\mT} - \pLdtdot,
	\end{equation}
	%
%	\begin{equation}
%		\frac{d}{dt}\LP\widehat{\dot{\bs{\zeta}}}|_{\f=\fd^{\|}}\RP_{\text{est}} = \epsilon^2( {\bf I} + {\bf K_v}{\bf K_p} )\zvdothat
%		+ \epsilon( {\bf K_p} + {\bf K_v} )\frac{d}{dt}\LP\zvdothatdot|_{\f=\fd^{\|}}\RP_{\text{est}},
%	\end{equation}
	%
	\begin{equation}
		\frac{d}{dt}\LP\dot{\Xiq}|_{\f=\fd^{\|}}\RP_{\text{est}} = \mT\left( {\bf K_p} \zvdothat
		+ {\bf K_v} \frac{d}{dt}\LP\zvdothatdot|_{\f=\fd^{\|}}\RP_{\text{est}}
		- \pLdqdot\right),
	\end{equation}
	%
	\begin{equation}
		\frac{d}{dt}\LP\qddothat|_{\f=\f_d^{\|}}\RP_{\text{est}} = \frac{1}{\|\Xiq\|^2}\LP\|\Xiq\|\LP\Skew{\qddothat}\Skew{\dird}+\Skew{\dird}\Skew{\qddothat}\RP-\frac{1}{\|\Xiq\|}\Skew[2]{\dird}\Xiq[T]\widehat{\dot{\Xiq}}\RP\widehat{\dot{\Xiq}}|_{\f=\fd^{\|}}+\frac{1}{\|\Xiq\|}\Skew[2]{\dird}\frac{d}{dt}\LP\widehat{\dot{\Xiq}}|_{\f=\fd^{\|}}\RP_{\text{est}},
	\end{equation}
	%
%	\begin{equation}
%		\wLdothat = \minus\frac{1}{\mQ\ell}\Skew{\dir}(\f+\widehat{\dist}),
%	\end{equation}
%	%
%	\begin{equation}
%		\dirddothat = \minus\Skew{\dir}\wLdothat - \|\wL\|^2\dir
%	\end{equation}
	%
	\begin{equation}
		\begin{aligned}
			\frac{d}{dt}\LP\frac{d}{dt}\LP\zvdothatdot|_{\f=\fd^{\|}}\RP_{\text{est}}|_{\f=\fd^{\|}}\RP_{\text{est}} =~& -\frac{1}{\mT}\left(\vphantom{\frac{d}{dt}\LP\widehat{\dot{\Xiq}}|_{\f=\fd^{\|}}\RP_{\text{est}}}\dot{\widehat{\bs{\varpi}}}\dir[T]\widehat{\Xiq}+\widehat{\bs{\varpi}}\widehat{\bs{\varpi}}\trs\widehat{\Xiq}+\widehat{\bs{\varpi}}\dir[T]\widehat{\dot{\Xiq}}+\widehat{\bs{\varpi}}\widehat{\bs{\varpi}}\trs\widehat{\Xiq}+\dir\dot{\widehat{\bs{\varpi}}}\trs\widehat{\Xiq}+\dir\widehat{\bs{\varpi}}\trs\widehat{\dot{\Xiq}}+\widehat{\bs{\varpi}}\dir[T]\widehat{\dot{\Xiq}}|_{\f=\fd^{\|}}\right.\nonumber\\
			&\left.+\dir\widehat{\bs{\varpi}}\trs\widehat{\dot{\Xiq}}|_{\f=\fd^{\|}}+\dir\dir[T]\frac{d}{dt}\LP\widehat{\dot{\Xiq}}|_{\f=\fd^{\|}}\RP_{\text{est}}\right)-\pLdqdot,
		\end{aligned}
	\end{equation}
	%
%	\begin{equation}
%		\frac{d}{dt}\LP\frac{d}{dt}\LP\widehat{\dot{\bs{\zeta}}}|_{\f=\fd^{\|}}\RP_{\text{est}}|_{\f=\fd^{\|}}\RP_{\text{est}} = \epsilon^2( {\bf I} + {\bf K_v}{\bf K_p} )\frac{d}{dt}\LP\zvdothatdot|_{\f=\fd^{\|}}\RP_{\text{est}}
%		+ \epsilon( {\bf K_p} + {\bf K_v} )\frac{d}{dt}\LP\frac{d}{dt}\LP\zvdothatdot|_{\f=\fd^{\|}}\RP_{\text{est}}|_{\f=\fd^{\|}}\RP_{\text{est}},
%	\end{equation}
	%
	\begin{equation}
		\frac{d}{dt}\LP\frac{d}{dt}\LP\widehat{\dot{\Xiq}}|_{\f=\fd^{\|}}\RP_{\text{est}}|_{\f=\fd^{\|}}\RP_{\text{est}} = \mT\left(  {\bf K_p} \frac{d}{dt}\LP\zvdothatdot|_{\f=\fd^{\|}}\RP_{\text{est}}
		+  {\bf K_v} \frac{d}{dt}\LP\frac{d}{dt}\LP\zvdothatdot|_{\f=\fd^{\|}}\RP_{\text{est}}|_{\f=\fd^{\|}}\RP_{\text{est}}
		- \pLd^{(5)}\right),
	\end{equation}
	%
	and, finally,
	%
	\begin{align}
		&\frac{d}{dt}\LP\widehat{\dot{\bs{\varpi}_d}}|_{\f=\fd^{\|}}\RP_{\text{est}}=-\frac{k_{\dir}}{h_{\dir}}\left(\Skew{\dot{\widehat{\bs{\varpi}}}}\Skew{\dir}\qdhat+\Skew{\widehat{\bs{\varpi}}}\Skew{\widehat{\bs{\varpi}}}\qdhat+\Skew{\widehat{\bs{\varpi}}}\Skew{\dir}	\frac{d}{dt}\LP\qdhat\RP_{\text{est}}+\Skew{\widehat{\bs{\varpi}}}\Skew{\widehat{\bs{\varpi}}}\qdhat+\Skew{\dir}\Skew{\dot{\widehat{\bs{\varpi}}}}\qdhat\right.\nonumber\\
		%
		&\left.+\Skew{\dir}\Skew{\widehat{\bs{\varpi}}}\frac{d}{dt}\LP\qdhat\RP_{\text{est}}+\Skew{\widehat{\bs{\varpi}}}\Skew{\dir}\qddothat|_{\f=\f_d^{\|}}+\Skew{\dir}\Skew{\widehat{\bs{\varpi}}}\qddothat|_{\f=\f_d^{\|}}+\Skew[2]{\dir}\frac{d}{dt}\LP\qddothat|_{\f=\f_d^{\|}}\RP_{\text{est}}\right) 
		-\left(\frac{1}{\|\widehat{\Xiq}\|}\Skew{\dot{\widehat{\bs{\varpi}}}}\Skew{\qdhat}\widehat{\dot{\Xiq}}|_{\f=\fd^{\|}}\right.\nonumber\\
		&\left.-\frac{1}{\|\widehat{\Xiq}\|}\Skew{\widehat{\bs{\varpi}}}\Skew{\frac{d}{dt}\LP\qdhat\RP_{\text{est}}}\widehat{\dot{\Xiq}}|_{\f=\fd^{\|}} - \frac{1}{\|\widehat{\Xiq}\|^2}\Skew{\widehat{\bs{\varpi}}}\Skew{\qdhat}\LP\|\widehat{\Xiq}\|\frac{d}{dt}\LP\widehat{\dot{\Xiq}}|_{\f=\fd^{\|}}\RP_{\text{est}}-\frac{1}{\|\widehat{\Xiq}\|}\widehat{\dot{\Xiq}}|_{\f=\fd^{\|}}\widehat{\Xiq}\trs\widehat{\dot{\Xiq}}\RP+\Skew{\widehat{\bs{\varpi}}}\Skew{\frac{\widehat{\dot{\Xiq}}|_{\f=\fd^{\|}}}{\|\widehat{\Xiq}\|}}\qddothat|_{\f=\f_d^{\|}}\right.\nonumber\\
		&\left.+\Skew{\dir}\Skew{\frac{\|\widehat{\Xiq}\|\frac{d}{dt}\LP\widehat{\dot{\Xiq}}|_{\f=\fd^{\|}}\RP_{\text{est}}-\frac{1}{\|\widehat{\Xiq}\|}\widehat{\dot{\Xiq}}|_{\f=\fd^{\|}}\widehat{\Xiq}\trs\widehat{\dot{\Xiq}}}{\|\widehat{\Xiq}\|^2}}\qddothat|_{\f=\f_d^{\|}}+\Skew{\dir}\Skew{\frac{\widehat{\dot{\Xiq}}|_{\f=\fd^{\|}}}{\|\widehat{\Xiq}\|}}\frac{d}{dt}\LP\qddothat|_{\f=\f_d^{\|}}\RP_{\text{est}} - \LP\Skew{\widehat{\bs{\varpi}}}\Skew{\qdhat} + \Skew{\dir}\Skew{\frac{d}{dt}\LP\qdhat\RP_{\text{est}}}\RP \right.\nonumber\\
		%
		&\left.-\Skew{\dir}\Skew{\qdhat}\left[\frac{1}{\|\widehat{\Xiq}\|^2}\left(\|\widehat{\Xiq}\|\frac{d}{dt}\LP\frac{d}{dt}\LP\widehat{\dot{\Xiq}}|_{\f=\fd^{\|}}\RP_{\text{est}}|_{\f=\fd^{\|}}\RP_{\text{est}}\right.\right.\right.\nonumber\\
		&\left.\left.\left.-\frac{1}{\|\widehat{\Xiq}\|}\frac{d}{dt}\LP\widehat{\dot{\Xiq}}|_{\f=\fd^{\|}}\RP_{\text{est}}|_{\f=\fd^{\|}}\widehat{\Xiq}\trs\widehat{\dot{\Xiq}}\right)-\frac{1}{\|\widehat{\Xiq}\|^3}\LP\frac{d}{dt}\widehat{\dot{\Xiq}}|_{\f=\fd^{\|}}\RP_{\text{est}}\widehat{\Xiq}\trs\widehat{\dot{\Xiq}}|_{\f=\fd^{\|}}-\frac{1}{\|\widehat{\Xiq}\|^6}\widehat{\dot{\Xiq}}|_{\f=\fd^{\|}}\LP\|\widehat{\Xiq}\|^3\widehat{\dot{\Xiq}}\trs-3\|\widehat{\Xiq}\|\widehat{\Xiq}\trs(\widehat{\Xiq}\trs\widehat{\dot{\Xiq}})\RP\widehat{\dot{\Xiq}}|_{\f=\fd^{\|}}\right.\right.\nonumber\\
		&\left.\left.+\frac{1}{\|\widehat{\Xiq}\|^3}\widehat{\dot{\Xiq}}|_{\f=\fd^{\|}}\widehat{\Xiq}\trs\frac{d}{dt}\LP\widehat{\dot{\Xiq}}|_{\f=\fd^{\|}}\RP_{\text{est}}\right]\right).
	\end{align} 

\section{Estimation Errors Associated term}

Based on auxiliary calculation in Section~\ref{sec:aux_notation}, the estimation related terms $\Psi_1$ in (25),  $\Psi_2$ in (27), $\Psi_3$ in (28) and $\Psi_4$ in (34) are expressed by
%
\begin{equation}
	\Psi_1:=\frac{\partial V_1}{\partial \vL}\LP\widetilde{\f_v}^{\|} + \frac{\mQ\ellp}{\mT}\widetilde{\f_q}^{\|} + \frac{1}{\mT}\LP\frac{\partial\Xiq}{\partial\vL}\RP^{\|}\vLtilde\RP \in \mathbb{R},
\end{equation}
%By exploiting the proposed KF, exchange the unmeasured linear velocity $\vL$ in \eqref{eq:zv} with estimation value which is obtained from \eqref{def:x_hat} as
%$\zv = \zvhat + \vLtilde$,
%%
%where $\zvhat:=\vLhat - \pLddot\in\mathbb{R}^3$. 
%Then
%%
%$ \Xiq = \widehat{\Xiq} + \partial \Xiq/\partial \zv\vLtilde$,
%%
%where $\widehat{\Xiq}:=\mT\LP{\bf K_p}\zp + {\bf K_v} \zvhat + g\ez - \pLdddot\RP$ and $\partial \Xiq/\partial \zv:= \mT{\bf K_v}$.
%
%where $\partial V_2/\partial \f_v=\frac{\partial V_1\trs}{\partial \vL}\dir\dir[T]$, $\partial V_2/\partial \f_q:=\mQ\ellp/\mT{\bf e}\trs\dir\dir[T]$ and $\partial V_2/\partial \zv:=1/\mT{\bf e}\trs\dir\dir[T]\partial \Xiq/\partial \zv$.
%
\begin{align}
	\Psi_2 := \Psi_1 - \frac{h_\v{q}}{\|\Xiq\|}\dir[T]\Skew[2]{\dird}  \frac{\partial \Xiq}{\partial \vL}\LP\widetilde{\f_v} 
	+ \frac{\mQ\ellp}{\mT}\widetilde{\f_q} \RP^{\|}\in\mathbb{R},
\end{align}
%
\begin{align}
	\Psi_3 := \Psi_2 + h_{\bs\varpi}\zc[T]\LP\widetilde{\f_q} - \LP\dot{\bs{\varpi_d}} - \widehat{\dot{\bs{\varpi_d}}}|_{\f=\f_d^{\|}}\RP -  \frac{h_\v{q}}{h_{\bs{\varpi}}}\LP\dird-\qdhat
	\RP + \frac{k_{\bs{\varpi}}}{h_{\bs{\varpi}}}\LP\zc-\zchat\RP\RP\in\mathbb{R},~\text{and}
\end{align}
%
\begin{align}
	\Psi_4:=\Psi_3 + h_\v{r}\rzd\trs \RotM\Skew{\ez}\LP
	\vphantom{\LP\frac{\partial V_4}{\partial\vL}\dir\RP\trs}
	-\RotM[T]\Skew{\rzd}(\rzddot-\widehat{\rzddot})
	-\frac{T_d}{h_\v{r}\mT}\Skew{\ez}\RotM[T](\bs{\delta}_2\trs-\widehat{\bs{\delta}_2}\trs)^{\|}
	+\frac{h_{\bs\varpi}}{h_\v{r}}\frac{T_d}{\mQ\ellp}\Skew{\ez}\RotM[T](\zc-\widehat{\zc})^{\perp}\RP
\end{align}
%
, respectively.

Using Young's inequality, $\Psi_1$ can be rewritten as
% 
\begin{equation}
	\begin{aligned}
	\Psi_1=\frac{\partial V_1}{\partial \vL}\LP\widetilde{\f_v}^{\|} + \frac{\mQ\ellp}{\mT}\widetilde{\f_q}^{\|} + \frac{1}{\mT}\LP\frac{\partial\Xiq}{\partial\vL}\RP^{\|}\vLtilde\RP&\leq \frac{\gamma_1}{2}(\beta\zp[T]+\zv[T])(\beta\zp+\zv) + \frac{1}{2\gamma_1}\delta_{\Psi_1}\\
	&\leq \frac{\gamma_1}{2} \beta^2\|\zp\|^2 +\frac{\gamma_1}{2}\|\zv\|^2 + \frac{1}{2\gamma_1}\delta_{\Psi_1}
    \end{aligned}
\end{equation}
%
where $\gamma_1>0$ and $\delta_{\Psi_1}:=\left\|\widetilde{\f_v}^{\|} + \frac{\mQ\ellp}{\mT}\widetilde{\f_q}^{\|} + \frac{1}{\mT}\LP\frac{\partial\Xiq}{\partial\vL}\RP^{\|}\vLtilde\right\|^2$. 
According to Section~\ref{sec:x_tilde}, $\widetilde{\bs{x}}$ is bounded, hence $\vLtilde$, $\widetilde{\bs{\varpi}}$, $\widetilde{\f_v}$ and $\widetilde{\f_q}$ are also bounded. 
In this case, $\delta_{\Psi_1}$ is bounded as well.

Then using Young's inequality, $\Psi_2$ can be rewritten as
%
\begin{equation}
	\begin{aligned}
		\Psi_2=\Psi_1 - \frac{h_\v{q}}{\|\Xiq\|}\dir[T]\Skew[2]{\dird}  \frac{\partial \Xiq}{\partial \vL}\LP\widetilde{\f_v} 
		+ \frac{\mQ\ellp}{\mT}\widetilde{\f_q} \RP^{\|}
%		&\leq \Psi_1 + \frac{\gamma_2}{2}(\frac{h_\v{q}}{\|\Xiq\|}\dir[T]\Skew[2]{\dird})\trs
%		(\frac{h_\v{q}}{\|\Xiq\|}\dir[T]\Skew[2]{\dird}) + \frac{1}{2\gamma_2}\delta_{\Psi_2}\\
		\leq \Psi_1 + \frac{\gamma_2}{2} \frac{h_\v{q}}{\|\Xiq\|} \|\dir[T]\Skew[2]{\dird}\|^2 + \frac{1}{2\gamma_2}\delta_{\Psi_2}
	\end{aligned}
\end{equation}
% 
where $\gamma_2>0$ and $\delta_{\Psi_2}:=\left\|\frac{\partial \Xiq}{\partial \vL}\LP\widetilde{\f_v} 
+ \frac{\mQ\ellp}{\mT}\widetilde{\f_q} \RP^{\|}\right\|^2$. 
According to $\widetilde{\f_v}$ and $\widetilde{\f_q}$ are bounded, $\delta_{\Psi_2}$ is bounded.

Using Young's inequality for $\Psi_3$, it can be formulated as
%
\begin{equation}
	\begin{aligned}
		\Psi_3=\Psi_2 + h_{\bs\varpi}\zc[T]\LP\widetilde{\f_q} - \LP\dot{\bs{\varpi_d}} - \widehat{\dot{\bs{\varpi_d}}}|_{\f=\f_d^{\|}}\RP -  \frac{h_\v{q}}{h_{\bs{\varpi}}}\LP\dird-\qdhat
		\RP + \frac{k_{\bs{\varpi}}}{h_{\bs{\varpi}}}\LP\zc-\zchat\RP\RP
		\leq \Psi_2 + \frac{\gamma_3}{2} \|\zc\|^2 + \frac{1}{2\gamma_3}\delta_{\Psi_3}
	\end{aligned}
\end{equation}
%
where $\gamma_3>0$ and $\delta_{\Psi_3}:=\left\|\widetilde{\f_q} - \LP\dot{\bs{\varpi_d}} - \widehat{\dot{\bs{\varpi_d}}}|_{\f=\f_d^{\|}}\RP -  \frac{h_\v{q}}{h_{\bs{\varpi}}}\LP\dird-\qdhat
\RP + \frac{k_{\bs{\varpi}}}{h_{\bs{\varpi}}}\LP\zc-\zchat\RP\right\|^2$. 
Due to $\vLtilde$ is bounded, then $\Xiq-\widehat{\Xiq}$ and $\dird-\qdhat$ are bounded.
Then based on the reference trajectory described in Section \uppercase\expandafter{\romannumeral4} \cite{2024_Yang_TIE} are bounded by construction, $\dot{\Xiq}|_{\f=\fd^{\|}}-\widehat{\dot{\Xiq}}|_{\f=\fd^{\|}}$, $\qddot|_{\f=\f_d^{\|}}-\qddothat|_{\f=\f_d^{\|}}$ and $\frac{d}{dt}\LP\widehat{\dot{\Xiq}}|_{\f=\fd^{\|}}\RP|_{\f=\fd^{\|}}-\frac{d}{dt}\LP\widehat{\dot{\Xiq}}|_{\f=\fd^{\|}}\RP_{\text{est}}|_{\f=\fd^{\|}}$ are bounded. 
Moreover, on account of $\widetilde{\bs{x}}$ is bounded, $\dot{\bs{\varpi_d}} - \widehat{\dot{\bs{\varpi_d}}}|_{\f=\f_d^{\|}}$ and $\zc-\zchat$ are bounded.
In this case, $\delta_{\Psi_3}$ is bounded.

%Using Young's inequality, $\Psi_4$ is restricted as
%%
%\begin{equation}
%	\begin{aligned}
%		\Psi_4&=\Psi_3 + h_\v{r}\rzd\trs \RotM\Skew{\ez}\LP
%		\vphantom{\LP\frac{\partial V_4}{\partial\vL}\dir\RP\trs}
%		-\RotM[T]\Skew{\rzd}(\rzddot-\widehat{\rzddot})
%		-\frac{T_d}{h_\v{r}\mT}\Skew{\ez}\RotM[T](\bs{\delta}_2\trs-\widehat{\bs{\delta}_2}\trs)^{\|}
%		+\frac{h_{\bs\varpi}}{h_\v{r}}\frac{T_d}{\mQ\ellp}\Skew{\ez}\RotM[T](\zc-\widehat{\zc})^{\perp}\RP\\
%		&\leq \Psi_3 + \frac{\gamma_4}{2} \|\zc\|^2 + \frac{1}{2\gamma_4}\delta_{\Psi_4}
%	\end{aligned}
%\end{equation}
%%
%where $\gamma_4>0$ and $\delta_{\Psi_4}:=$. 
Similarly, $\rzddot-\widehat{\rzddot}$ and $\bs{\delta}_2\trs-\widehat{\bs{\delta}_2}\trs$ are also bounded.
Due to Rotation matrix $\RotM$ property and, $\rz$ and $\rzd$ are unit vector, $\delta_{\Psi_4}:=h_\v{r}\rzd\trs \RotM\Skew{\ez}\LP
\vphantom{\LP\frac{\partial V_4}{\partial\vL}\dir\RP\trs}
-\RotM[T]\Skew{\rzd}(\rzddot-\widehat{\rzddot})
-\frac{T_d}{h_\v{r}\mT}\Skew{\ez}\RotM[T](\bs{\delta}_2\trs-\widehat{\bs{\delta}_2}\trs)^{\|}
+\frac{h_{\bs\varpi}}{h_\v{r}}\frac{T_d}{\mQ\ellp}\Skew{\ez}\RotM[T](\zc-\widehat{\zc})^{\perp}\RP$ is bounded as well.
And all of the estimation error related term that remained in time derivative of Lyapunov function $V_4$ is introduced as
%
\begin{equation}
	\delta_{V_4} := \delta_{\Psi_1} + \delta_{\Psi_2} + \delta_{\Psi_3} + \delta_{\Psi_4}.
\end{equation}
%
In turn, the boundedness of $\delta_{V_4}$ is confirmed.
Then the time derivative of the fourth Lyapunov function candidate $V_4$ can be expressed as
%

\begin{equation}
	\dot{V}_4 = -(\v{P}\v{z})\trs\v{J}\overline{\v{Q}}^{-1}\overline{\v{Q}}(\v{P}\v{z}) + \Psi_4
	\leq
	-(\v{P}\v{z})\trs\v{J}^{\star}\overline{\v{Q}}^{-1}\overline{\v{Q}}(\v{P}\v{z}) + \delta_{V_4}
	\leq -\frac{\lambda_{\min}(\v{J}^{\star})}{\lambda_{\max}(\overline{\v{Q}})}V_4+ \delta_{V_4}.
\end{equation}
%
where
%
\begin{align}
	\v{J}^{\star} := \begin{bmatrix}
		\beta{\bf K_p}-\frac{\gamma_1}{2} \beta^2\v{I} & \frac{\beta}{2}{\bf K_v}-\frac{\gamma_1}{4}\v{I} & -\frac{\beta\|\Xiq\|}{2\mT}\v{I} & \v{0} & \v{0} \\
		\frac{\beta}{2}{\bf K_v} - \frac{\gamma_1}{4}\v{I}& -(\beta-{\bf K_v}) & -\frac{\|\Xiq\|}{2\mT}\v{I} & \v{0} & \v{0}  \\
		-\frac{\beta\|\Xiq\|}{2\mT}\v{I} & -\frac{\|\Xiq\|}{2\mT}\v{I} & (k_\v{q}-\frac{\gamma_2}{2} \frac{h_\v{q}}{\|\Xiq\|})\v{I} & \v{0} & \v{0} \\
		\v{0} & \v{0} & \v{0} & (k_{\bs{\varpi}}-\frac{\gamma_3}{2})\v{I} & \v{0}\\
		\v{0} & \v{0} & \v{0} & \v{0} & k_\v{r}\v{I} \\
	\end{bmatrix} \nonumber
\end{align}

\begin{remark}
	$\gamma_1$, $\gamma_2$ and $\gamma_3$ are simple analysis parameters that can always be adjusted to render $\v{J}^{\star}$ positive definite and they do not play any role whatsoever in the controller performance.
\end{remark}

\begin{thebibliography}{9}
\bibitem{2024_Yang_TIE}
Yang, W., Reis, J., Yu, G. and Silvestre, C. Nonlinear Output Feedback Control of an Underactuated Inverted Pendulum. {\em IEEE Transactions on Industrial Electronics}. pp. 1-8 (2024)
\bibitem{2002_Khalil}
Khalil, H. K.. Pearson education, Nonlinear systems (3rd ed.). Upper Saddle River, New Jersey: Prentice Hall. (2002)
\end{thebibliography}

\end{document}


